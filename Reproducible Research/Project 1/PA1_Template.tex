% Options for packages loaded elsewhere
\PassOptionsToPackage{unicode}{hyperref}
\PassOptionsToPackage{hyphens}{url}
%
\documentclass[
]{article}
\usepackage{lmodern}
\usepackage{amssymb,amsmath}
\usepackage{ifxetex,ifluatex}
\ifnum 0\ifxetex 1\fi\ifluatex 1\fi=0 % if pdftex
  \usepackage[T1]{fontenc}
  \usepackage[utf8]{inputenc}
  \usepackage{textcomp} % provide euro and other symbols
\else % if luatex or xetex
  \usepackage{unicode-math}
  \defaultfontfeatures{Scale=MatchLowercase}
  \defaultfontfeatures[\rmfamily]{Ligatures=TeX,Scale=1}
\fi
% Use upquote if available, for straight quotes in verbatim environments
\IfFileExists{upquote.sty}{\usepackage{upquote}}{}
\IfFileExists{microtype.sty}{% use microtype if available
  \usepackage[]{microtype}
  \UseMicrotypeSet[protrusion]{basicmath} % disable protrusion for tt fonts
}{}
\makeatletter
\@ifundefined{KOMAClassName}{% if non-KOMA class
  \IfFileExists{parskip.sty}{%
    \usepackage{parskip}
  }{% else
    \setlength{\parindent}{0pt}
    \setlength{\parskip}{6pt plus 2pt minus 1pt}}
}{% if KOMA class
  \KOMAoptions{parskip=half}}
\makeatother
\usepackage{xcolor}
\IfFileExists{xurl.sty}{\usepackage{xurl}}{} % add URL line breaks if available
\IfFileExists{bookmark.sty}{\usepackage{bookmark}}{\usepackage{hyperref}}
\hypersetup{
  pdftitle={PA1\_Template.R},
  pdfauthor={drall},
  hidelinks,
  pdfcreator={LaTeX via pandoc}}
\urlstyle{same} % disable monospaced font for URLs
\usepackage[margin=1in]{geometry}
\usepackage{color}
\usepackage{fancyvrb}
\newcommand{\VerbBar}{|}
\newcommand{\VERB}{\Verb[commandchars=\\\{\}]}
\DefineVerbatimEnvironment{Highlighting}{Verbatim}{commandchars=\\\{\}}
% Add ',fontsize=\small' for more characters per line
\usepackage{framed}
\definecolor{shadecolor}{RGB}{248,248,248}
\newenvironment{Shaded}{\begin{snugshade}}{\end{snugshade}}
\newcommand{\AlertTok}[1]{\textcolor[rgb]{0.94,0.16,0.16}{#1}}
\newcommand{\AnnotationTok}[1]{\textcolor[rgb]{0.56,0.35,0.01}{\textbf{\textit{#1}}}}
\newcommand{\AttributeTok}[1]{\textcolor[rgb]{0.77,0.63,0.00}{#1}}
\newcommand{\BaseNTok}[1]{\textcolor[rgb]{0.00,0.00,0.81}{#1}}
\newcommand{\BuiltInTok}[1]{#1}
\newcommand{\CharTok}[1]{\textcolor[rgb]{0.31,0.60,0.02}{#1}}
\newcommand{\CommentTok}[1]{\textcolor[rgb]{0.56,0.35,0.01}{\textit{#1}}}
\newcommand{\CommentVarTok}[1]{\textcolor[rgb]{0.56,0.35,0.01}{\textbf{\textit{#1}}}}
\newcommand{\ConstantTok}[1]{\textcolor[rgb]{0.00,0.00,0.00}{#1}}
\newcommand{\ControlFlowTok}[1]{\textcolor[rgb]{0.13,0.29,0.53}{\textbf{#1}}}
\newcommand{\DataTypeTok}[1]{\textcolor[rgb]{0.13,0.29,0.53}{#1}}
\newcommand{\DecValTok}[1]{\textcolor[rgb]{0.00,0.00,0.81}{#1}}
\newcommand{\DocumentationTok}[1]{\textcolor[rgb]{0.56,0.35,0.01}{\textbf{\textit{#1}}}}
\newcommand{\ErrorTok}[1]{\textcolor[rgb]{0.64,0.00,0.00}{\textbf{#1}}}
\newcommand{\ExtensionTok}[1]{#1}
\newcommand{\FloatTok}[1]{\textcolor[rgb]{0.00,0.00,0.81}{#1}}
\newcommand{\FunctionTok}[1]{\textcolor[rgb]{0.00,0.00,0.00}{#1}}
\newcommand{\ImportTok}[1]{#1}
\newcommand{\InformationTok}[1]{\textcolor[rgb]{0.56,0.35,0.01}{\textbf{\textit{#1}}}}
\newcommand{\KeywordTok}[1]{\textcolor[rgb]{0.13,0.29,0.53}{\textbf{#1}}}
\newcommand{\NormalTok}[1]{#1}
\newcommand{\OperatorTok}[1]{\textcolor[rgb]{0.81,0.36,0.00}{\textbf{#1}}}
\newcommand{\OtherTok}[1]{\textcolor[rgb]{0.56,0.35,0.01}{#1}}
\newcommand{\PreprocessorTok}[1]{\textcolor[rgb]{0.56,0.35,0.01}{\textit{#1}}}
\newcommand{\RegionMarkerTok}[1]{#1}
\newcommand{\SpecialCharTok}[1]{\textcolor[rgb]{0.00,0.00,0.00}{#1}}
\newcommand{\SpecialStringTok}[1]{\textcolor[rgb]{0.31,0.60,0.02}{#1}}
\newcommand{\StringTok}[1]{\textcolor[rgb]{0.31,0.60,0.02}{#1}}
\newcommand{\VariableTok}[1]{\textcolor[rgb]{0.00,0.00,0.00}{#1}}
\newcommand{\VerbatimStringTok}[1]{\textcolor[rgb]{0.31,0.60,0.02}{#1}}
\newcommand{\WarningTok}[1]{\textcolor[rgb]{0.56,0.35,0.01}{\textbf{\textit{#1}}}}
\usepackage{graphicx,grffile}
\makeatletter
\def\maxwidth{\ifdim\Gin@nat@width>\linewidth\linewidth\else\Gin@nat@width\fi}
\def\maxheight{\ifdim\Gin@nat@height>\textheight\textheight\else\Gin@nat@height\fi}
\makeatother
% Scale images if necessary, so that they will not overflow the page
% margins by default, and it is still possible to overwrite the defaults
% using explicit options in \includegraphics[width, height, ...]{}
\setkeys{Gin}{width=\maxwidth,height=\maxheight,keepaspectratio}
% Set default figure placement to htbp
\makeatletter
\def\fps@figure{htbp}
\makeatother
\setlength{\emergencystretch}{3em} % prevent overfull lines
\providecommand{\tightlist}{%
  \setlength{\itemsep}{0pt}\setlength{\parskip}{0pt}}
\setcounter{secnumdepth}{-\maxdimen} % remove section numbering

\title{PA1\_Template.R}
\author{drall}
\date{2020-07-19}

\begin{document}
\maketitle

\begin{Shaded}
\begin{Highlighting}[]
\CommentTok{## RR-project 1 - Abdou Allayeh}
\CommentTok{## Data loading}

\KeywordTok{setwd}\NormalTok{(}\StringTok{"~/RDataScience/Reproducible Research/Project 1"}\NormalTok{)}
\KeywordTok{download.file}\NormalTok{(}\StringTok{"https://d396qusza40orc.cloudfront.net/repdata%2Fdata%2Factivity.zip"}\NormalTok{, }\DataTypeTok{destfile =} \StringTok{"activity.zip"}\NormalTok{, }\DataTypeTok{mode=}\StringTok{"wb"}\NormalTok{)}

\CommentTok{## unzip data and read }

\KeywordTok{unzip}\NormalTok{(}\StringTok{"activity.zip"}\NormalTok{)}
\NormalTok{Rawdata <-}\StringTok{ }\KeywordTok{read.csv}\NormalTok{(}\StringTok{"activity.csv"}\NormalTok{, }\DataTypeTok{header =} \OtherTok{TRUE}\NormalTok{)}
\KeywordTok{head}\NormalTok{(Rawdata)}
\end{Highlighting}
\end{Shaded}

\begin{verbatim}
##   steps       date interval
## 1    NA 2012-10-01        0
## 2    NA 2012-10-01        5
## 3    NA 2012-10-01       10
## 4    NA 2012-10-01       15
## 5    NA 2012-10-01       20
## 6    NA 2012-10-01       25
\end{verbatim}

\begin{Shaded}
\begin{Highlighting}[]
\CommentTok{## Calculate total number of steps per day}

\NormalTok{main_data <-}\StringTok{ }\KeywordTok{na.omit}\NormalTok{(Rawdata)}
\NormalTok{steps_per_day <-}\StringTok{ }\KeywordTok{aggregate}\NormalTok{(main_data}\OperatorTok{$}\NormalTok{steps, }\DataTypeTok{by =} \KeywordTok{list}\NormalTok{(}\DataTypeTok{Steps.Date =}\NormalTok{ main_data}\OperatorTok{$}\NormalTok{date), }\DataTypeTok{FUN =} \StringTok{"sum"}\NormalTok{)}

\KeywordTok{hist}\NormalTok{(steps_per_day}\OperatorTok{$}\NormalTok{x, }\DataTypeTok{col =} \StringTok{"gold"}\NormalTok{, }
     \DataTypeTok{breaks =} \DecValTok{20}\NormalTok{,}
     \DataTypeTok{main =} \StringTok{"Histogram for Total steps per day"}\NormalTok{,}
     \DataTypeTok{xlab =} \StringTok{"Total steps per day"}\NormalTok{)}
\end{Highlighting}
\end{Shaded}

\includegraphics{PA1_Template_files/figure-latex/unnamed-chunk-1-1.pdf}

\begin{Shaded}
\begin{Highlighting}[]
\CommentTok{## Calculate the mean and median of all steps per day}

\NormalTok{main_mean <-}\StringTok{ }\KeywordTok{mean}\NormalTok{(steps_per_day[,}\DecValTok{2}\NormalTok{])}
\KeywordTok{print}\NormalTok{(main_mean)}
\end{Highlighting}
\end{Shaded}

\begin{verbatim}
## [1] 10766.19
\end{verbatim}

\begin{Shaded}
\begin{Highlighting}[]
\NormalTok{main_median <-}\StringTok{ }\KeywordTok{median}\NormalTok{(steps_per_day[,}\DecValTok{2}\NormalTok{])}
\KeywordTok{print}\NormalTok{ (main_median)}
\end{Highlighting}
\end{Shaded}

\begin{verbatim}
## [1] 10765
\end{verbatim}

\begin{Shaded}
\begin{Highlighting}[]
\CommentTok{## average activity pattern per day}
\CommentTok{## Time Plotting for all steps per averaged of all days, along all 5-min intervals}

\NormalTok{avaraged_day <-}\StringTok{ }\KeywordTok{aggregate}\NormalTok{(main_data}\OperatorTok{$}\NormalTok{steps, }
                          \DataTypeTok{by =} \KeywordTok{list}\NormalTok{(}\DataTypeTok{Interval =}\NormalTok{ main_data}\OperatorTok{$}\NormalTok{interval), }
                          \DataTypeTok{FUN =} \StringTok{"mean"}\NormalTok{)}
\KeywordTok{plot}\NormalTok{(avaraged_day}\OperatorTok{$}\NormalTok{Interval, avaraged_day}\OperatorTok{$}\NormalTok{x, }\DataTypeTok{type =} \StringTok{"l"}\NormalTok{, }
     \DataTypeTok{main =} \StringTok{"Average activity pattern per day"}\NormalTok{, }
     \DataTypeTok{ylab =} \StringTok{"Avarage number of steps "}\NormalTok{, }
     \DataTypeTok{xlab =} \StringTok{"5-min intervals"}\NormalTok{)}
\end{Highlighting}
\end{Shaded}

\includegraphics{PA1_Template_files/figure-latex/unnamed-chunk-1-2.pdf}

\begin{Shaded}
\begin{Highlighting}[]
\CommentTok{## define the interval with the maximum number of steps}

\NormalTok{interval_row <-}\StringTok{ }\KeywordTok{which.max}\NormalTok{(avaraged_day}\OperatorTok{$}\NormalTok{x)}
\NormalTok{max_interval <-}\StringTok{ }\NormalTok{avaraged_day[interval_row,}\DecValTok{1}\NormalTok{]}
\KeywordTok{print}\NormalTok{ (max_interval)}
\end{Highlighting}
\end{Shaded}

\begin{verbatim}
## [1] 835
\end{verbatim}

\begin{Shaded}
\begin{Highlighting}[]
\CommentTok{## calculate the total number of NA values}

\NormalTok{NA_number <-}\StringTok{ }\KeywordTok{length}\NormalTok{(}\KeywordTok{which}\NormalTok{(}\KeywordTok{is.na}\NormalTok{(Rawdata}\OperatorTok{$}\NormalTok{steps)))}
\KeywordTok{print}\NormalTok{ (NA_number)}
\end{Highlighting}
\end{Shaded}

\begin{verbatim}
## [1] 2304
\end{verbatim}

\begin{Shaded}
\begin{Highlighting}[]
\CommentTok{## Histogram for new frequencies of all steps}

\NormalTok{missingVals <-}\StringTok{ }\KeywordTok{sum}\NormalTok{(}\KeywordTok{is.na}\NormalTok{(data))}
\end{Highlighting}
\end{Shaded}

\begin{verbatim}
## Warning in is.na(data): is.na() applied to non-(list or vector) of type
## 'closure'
\end{verbatim}

\begin{Shaded}
\begin{Highlighting}[]
\KeywordTok{library}\NormalTok{(magrittr)}
\KeywordTok{library}\NormalTok{(dplyr)}
\end{Highlighting}
\end{Shaded}

\begin{verbatim}
## 
## Attaching package: 'dplyr'
\end{verbatim}

\begin{verbatim}
## The following objects are masked from 'package:stats':
## 
##     filter, lag
\end{verbatim}

\begin{verbatim}
## The following objects are masked from 'package:base':
## 
##     intersect, setdiff, setequal, union
\end{verbatim}

\begin{Shaded}
\begin{Highlighting}[]
\NormalTok{replacewithmean <-}\StringTok{ }\ControlFlowTok{function}\NormalTok{(x) }\KeywordTok{replace}\NormalTok{(x, }\KeywordTok{is.na}\NormalTok{(x), }\KeywordTok{mean}\NormalTok{(x, }\DataTypeTok{na.rm =} \OtherTok{TRUE}\NormalTok{))}
\NormalTok{meandata <-}\StringTok{ }\NormalTok{Rawdata }\OperatorTok\StringTok{ }\KeywordTok{group_by}\NormalTok{(interval) }\OperatorTok\StringTok{ }\KeywordTok{mutate}\NormalTok{(}\DataTypeTok{steps =} \KeywordTok{replacewithmean}\NormalTok{(steps))}
\KeywordTok{head}\NormalTok{(meandata)}
\end{Highlighting}
\end{Shaded}

\begin{verbatim}
## # A tibble: 6 x 3
## # Groups:   interval [6]
##    steps date       interval
##    <dbl> <chr>         <int>
## 1 1.72   2012-10-01        0
## 2 0.340  2012-10-01        5
## 3 0.132  2012-10-01       10
## 4 0.151  2012-10-01       15
## 5 0.0755 2012-10-01       20
## 6 2.09   2012-10-01       25
\end{verbatim}

\begin{Shaded}
\begin{Highlighting}[]
\CommentTok{## Histogram of all steps per day }
\CommentTok{## Calculate and report the mean and median all steps per day}

\NormalTok{FullSummedDataByDay <-}\StringTok{ }\KeywordTok{aggregate}\NormalTok{(meandata}\OperatorTok{$}\NormalTok{steps, }\DataTypeTok{by=}\KeywordTok{list}\NormalTok{(meandata}\OperatorTok{$}\NormalTok{date), sum)}

\KeywordTok{names}\NormalTok{(FullSummedDataByDay)[}\DecValTok{1}\NormalTok{] =}\StringTok{"date"}
\KeywordTok{names}\NormalTok{(FullSummedDataByDay)[}\DecValTok{2}\NormalTok{] =}\StringTok{"totalsteps"}
\KeywordTok{head}\NormalTok{(FullSummedDataByDay,}\DecValTok{15}\NormalTok{)}
\end{Highlighting}
\end{Shaded}

\begin{verbatim}
##          date totalsteps
## 1  2012-10-01   10766.19
## 2  2012-10-02     126.00
## 3  2012-10-03   11352.00
## 4  2012-10-04   12116.00
## 5  2012-10-05   13294.00
## 6  2012-10-06   15420.00
## 7  2012-10-07   11015.00
## 8  2012-10-08   10766.19
## 9  2012-10-09   12811.00
## 10 2012-10-10    9900.00
## 11 2012-10-11   10304.00
## 12 2012-10-12   17382.00
## 13 2012-10-13   12426.00
## 14 2012-10-14   15098.00
## 15 2012-10-15   10139.00
\end{verbatim}

\begin{Shaded}
\begin{Highlighting}[]
\CommentTok{## Summary of newest data and making histogram}

\KeywordTok{summary}\NormalTok{(FullSummedDataByDay)}
\end{Highlighting}
\end{Shaded}

\begin{verbatim}
##      date             totalsteps   
##  Length:61          Min.   :   41  
##  Class :character   1st Qu.: 9819  
##  Mode  :character   Median :10766  
##                     Mean   :10766  
##                     3rd Qu.:12811  
##                     Max.   :21194
\end{verbatim}

\begin{Shaded}
\begin{Highlighting}[]
\KeywordTok{hist}\NormalTok{(FullSummedDataByDay}\OperatorTok{$}\NormalTok{totalsteps, }\DataTypeTok{xlab =} \StringTok{"Steps"}\NormalTok{, }\DataTypeTok{ylab =} \StringTok{"Frequency"}\NormalTok{, }\DataTypeTok{main =} \StringTok{"Total Daily Steps"}\NormalTok{, }\DataTypeTok{breaks =} \DecValTok{20}\NormalTok{)}

\CommentTok{## Compare between the mean and median of Old and New data}

\NormalTok{oldmean <-}\StringTok{ }\NormalTok{main_mean}
\NormalTok{newmean <-}\StringTok{ }\KeywordTok{mean}\NormalTok{(FullSummedDataByDay}\OperatorTok{$}\NormalTok{totalsteps)}
\NormalTok{oldmedian <-}\StringTok{ }\NormalTok{main_median}
\NormalTok{newmedian <-}\StringTok{ }\KeywordTok{median}\NormalTok{(FullSummedDataByDay}\OperatorTok{$}\NormalTok{totalsteps)}

\CommentTok{## differences in activity patterns between weekdays and weekends}
\CommentTok{## Plotting for Comparison of average all steps per each interval}

\NormalTok{meandata}\OperatorTok{$}\NormalTok{date <-}\StringTok{ }\KeywordTok{as.Date}\NormalTok{(meandata}\OperatorTok{$}\NormalTok{date)}
\NormalTok{meandata}\OperatorTok{$}\NormalTok{weekday <-}\StringTok{ }\KeywordTok{weekdays}\NormalTok{(meandata}\OperatorTok{$}\NormalTok{date)}
\NormalTok{meandata}\OperatorTok{$}\NormalTok{weekend <-}\StringTok{ }\KeywordTok{ifelse}\NormalTok{(meandata}\OperatorTok{$}\NormalTok{weekday}\OperatorTok{==}\StringTok{"Saturday"} \OperatorTok{|}\StringTok{ }\NormalTok{meandata}\OperatorTok{$}\NormalTok{weekday}\OperatorTok{==}\StringTok{"Sunday"}\NormalTok{, }\StringTok{"Weekend"}\NormalTok{, }\StringTok{"Weekday"}\NormalTok{ )}

\KeywordTok{library}\NormalTok{(ggplot2)}
\end{Highlighting}
\end{Shaded}

\includegraphics{PA1_Template_files/figure-latex/unnamed-chunk-1-3.pdf}

\begin{Shaded}
\begin{Highlighting}[]
\NormalTok{meandataweekendweekday <-}\StringTok{ }\KeywordTok{aggregate}\NormalTok{(meandata}\OperatorTok{$}\NormalTok{steps , }\DataTypeTok{by=} \KeywordTok{list}\NormalTok{(meandata}\OperatorTok{$}\NormalTok{weekend, meandata}\OperatorTok{$}\NormalTok{interval), }\KeywordTok{na.omit}\NormalTok{(mean))}
\KeywordTok{names}\NormalTok{(meandataweekendweekday) <-}\StringTok{ }\KeywordTok{c}\NormalTok{(}\StringTok{"weekend"}\NormalTok{, }\StringTok{"interval"}\NormalTok{, }\StringTok{"steps"}\NormalTok{)}

\KeywordTok{ggplot}\NormalTok{(meandataweekendweekday, }\KeywordTok{aes}\NormalTok{(}\DataTypeTok{x=}\NormalTok{interval, }\DataTypeTok{y=}\NormalTok{steps, }\DataTypeTok{color=}\NormalTok{weekend)) }\OperatorTok{+}\StringTok{ }\KeywordTok{geom_line}\NormalTok{()}\OperatorTok{+}
\StringTok{  }\KeywordTok{facet_grid}\NormalTok{(weekend }\OperatorTok{~}\NormalTok{.) }\OperatorTok{+}\StringTok{ }\KeywordTok{xlab}\NormalTok{(}\StringTok{"Interval"}\NormalTok{) }\OperatorTok{+}\StringTok{ }\KeywordTok{ylab}\NormalTok{(}\StringTok{"Mean Steps"}\NormalTok{) }\OperatorTok{+}
\StringTok{  }\KeywordTok{ggtitle}\NormalTok{(}\StringTok{"Comparison of Average All Steps per each Interval"}\NormalTok{)}
\end{Highlighting}
\end{Shaded}

\includegraphics{PA1_Template_files/figure-latex/unnamed-chunk-1-4.pdf}

\end{document}
